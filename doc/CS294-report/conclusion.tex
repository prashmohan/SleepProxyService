\section{Conclusion and Future Work} % (fold)
\label{sec:conclusion_and_future_work}
In this report we motivated the need for energy proportional computing and the requirement of putting nodes to sleep in today's architecture. We present ``Clean Clusters'' a working implementation of a power aware cluster management suite. The suite contains many building blocks for the creation of more sophisticated and efficient job schedulers. Apart from using CC to create a gexec clone, we are also working on using CC to make parallel python and Torque/Maui power aware. We transfer our learning from using CC in a small cluster onto a real world HPC cluster trace. Evaluation of various sleep scheduling algorithms has provided with a model of the power-performance tradeoff for easily matching requirements with the tradeoff. 

No work is complete without a real world deployment and verification of results. Our immediate focus is on creating a release of CC plugged into Torque/Maui which can be deployed in a real cluster within Soda Hall and measuring its effectiveness. This would come with it's own set of learning and open up a set of opportunities. We are currently gathering data about job allocation as well as real time resource usage in nodes in order to improve our scheduling algorithms. Another matter of practical importance is to understand the electrical effects of such an approach; would the circuits be able to handle multiple machines switching on at the same time, are there inherent hardware limitations on the number of reboots, how reliable is IPMI and wake on lan, etc. We also need to create a ``smarter'' scheduler which will automatically adapt the slack set of active nodes based on history and adapt its tradeoff based on the changing cost of electricity, etc. The evaluation today also does not take into account the benefits of dynamic voltage and frequency scaling which could provide us with additional power saving. Our initial evaluation points us to believe that we can save more than 80\% of electricity in HPC clusters with very minimal effect on user experience and we hope to see the same in our deployment.
% section conclusion_and_future_work (end)